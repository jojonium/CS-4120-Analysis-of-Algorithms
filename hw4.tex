\documentclass[a4paper, 10pt]{article}

\usepackage[utf8]{inputenc} % UTF-8 support
\usepackage{amsmath}
\usepackage[]{algorithm2e}
\usepackage{biblatex}

\title{Analysis of Algorithms Homework 4}
\author{Joseph Petitti}
\date{\today}
\bibliography{hw4}

\begin{document}
\IncMargin{1em}

\maketitle

\begin{enumerate}
	\item 
		We assume $G$ does not contain a cycle where the total cost of all edges
		in the cycle is negative because the concept of a shortest path is
		meaningless in a graph with negative-weight cycles.

		\textbf{Algorithm}:

		\begin{enumerate}
			\item Remove $e$ from the graph, such that $G' = (V, E \; \backslash
				\; e)$. Use Dijkstra's algorithm to find the shortest path
				between $s$ and $t$ on $G'$ and store it as $S_1$.
			\item Let $u$ be the starting node of $e$. Use Dijkstra's algorithm
				to find the shortest path between $s$ and $u$ and store it as
				$S_2$.
			\item Let $v$ be the ending node of $e$. Use Dijkstra's algorithm to
				find the shortest path between $v$ and $t$ on $G$ and store it
				as $S_3$.
			\item Return $\min ( S_1, S_2 + \textrm{value}(e) + S_3 )$.
		\end{enumerate}

		\textbf{Analysis of Running Time}:

		The running time of Dijkstra's Algorithm is analyzed in ``Algorithm
		Design'' by Jon Kleinberg and Éva Tardos as $O(m \log n)$. The algorithm
		above merely uses Dijkstra's algorithm three times, giving it a time
		complexity of $O(3m \log n)$, which simplifies to $O(m \log n)$.

		\textbf{Proof of Correctness}:

		There are two possible scenarios: either $e$ is in the shortest path
		between $s$ and $t$, or it is not.

		\begin{itemize}
			\item If $e$ is not in the shortest path, step (a) above will
				correctly find the shortest by using Dijkstra's Algorithm on a
				graph without $e$. The correctness of Dijkstra's Algorithm is
				proven in ``Algorithm Design'' by Jon Kleinberg and Éva Tardos.
			\item If $e$ is in the shortest path, then we can split the shortest
				path into three segments: the path from $s$ to $u$, $e$, and the
				path from $v$ to $t$. Dijkstra's Algorithm correctly finds the
				solution to the first and third segments, and the second segment
				is simply the weight of $e$. Summing these correctly computed
				sub-problems will give us the correct optimal solution.
		\end{itemize}

	\item \textbf{Algorithm}:

		The Wagner--Fischer Algorithm is a dynamic programming algorithm that
		computes the edit distance between two strings, and has a running time
		of $O(m \times n)$.

		Let Array be a two-dimensional $(m+1)$-by-$(n+1)$ array filled with
		zeroes. For each $i, j$, $\textrm{Array}[i][j]$ will hold the edit
		distance between the first $i$ characters of $X$ and the first $j$
		characters of $Y$.

		\begin{algorithm}[h]
			\SetKwData{Array}{Array}
			\SetKwData{SubstitutionCost}{SubstitutionCost}
			\tcp{i deletions turns string of length i into empty string}
			\For{$i \gets 1$ \KwTo $m$}{
				$\Array[i][0] \gets i$ \;
			}
			\BlankLine
			\tcp{i insertions from empty string is string of length i}
			\For{$j \gets 1$ \KwTo $n$}{
				$\Array[0][j] \gets j$ \;
			}
			\BlankLine

			\For{$j \gets 1$ \KwTo $n$}{
				\For{$i \gets 1$ \KwTo $m$}{
					\eIf{$x_i = y_j$}{
						$\SubstitutionCost \gets 0$ \;
					}{
						$\SubstitutionCost \gets 1$ \;
					}
					$\Array[i][j] \gets \min($ \\
					\quad $\Array[i - 1][j] + 1, $ \tcp*[f]{deletion} \\
					\quad $\Array[i][j - 1] + 1, $ \tcp*[f]{insertion} \\
					\quad $\Array[i - 1][j - 1] + \SubstitutionCost$
					\tcp*[f]{substitution} \\
					$)$ \;
				}
			}

			\Return $\Array[m][n]$
			\label{alg:prob2}
		\end{algorithm}

		The edit distance problem is a special case of the sequence alignment
		problem because we are also allowed to make substitutions. In the
		algorithm above, the gap penalty $\delta$ is 1 and the mismatch penalty
		$\alpha$ is 1.

		\textbf{Analysis of Running Time}:

		The algorithm above contains two nested loops. The outer loop runs $n$
		times, and the inner loop runs $m$ times, giving us a running time of
		$O(n \times m)$.

		\textbf{Proof of Correctness by Induction}:

		\textbf{Base cases}: The algorithm above correctly returns the minimum
		edit distance for the following base cases:

		\begin{itemize}
			\item When $n = 0$, you can get from $X$ to $Y$ in $m$ steps simply
				by deleting all $m$ characters in $X$.
			\item When $m = 0$, you can get from $X$ to $Y$ in $n$ steps simply
				by inserting all $n$ characters from $Y$ into $X$.
		\end{itemize}

		\textbf{Inductive step}: At any $i, j$, there are only three possible
		operations you can perform:

		\begin{enumerate}
			\item Substitute $x_i$ for $y_j$. The total edit distance at this
				point is the edit distance between $x_1 x_2 \dots x_{i - 1}$ and
				$y_1 y_2 \dots y_{j-1}$ plus the cost of the substitution (if
				$x_i \ne y_j$). By the inductive assumption we have already
				calculated this previous value correctly.
			\item Delete $x_i$. This costs 1, so the total edit distance at this
				point is 1 plus the edit distance between $x_1 x_2 \dots
				x_{i-1}$ and $y_1 y_2 \dots y_{j}$, which we have already
				calculated by the inductive assumption.
			\item Insert $y_j$. This costs 1, so the total edit distance at this
				point is 1 plus the edit distance between $x_1 x_2 \dots
				x_{i}$ and $y_1 y_2 \dots y_{j - 1}$, which we have already
				calculated by the inductive assumption.
		\end{enumerate}

		Each of these three possible actions leads back to one of the base cases
		above, so the algorithm is correct by induction for any $i, j$.

	\item 
		\begin{enumerate}
			\item We need to check each edge in $C(S \; \backslash \; j, n)$ to
				see where we can insert a visit to $j$ for the minimum cost. Let
				V be the cities in $C(S \; \backslash \; j, n)$ in the
				order that they are visited, indexed starting at 1. Let Array be
				an empty array of length $n - 1$, indexed starting at 1. The
				cost change of inserting the visit to $j$ after the visit to
				city $i$ will be store in $\textrm{Array}[i]$.

				\IncMargin{2em}
				\begin{algorithm}[h]
					\SetKwData{Array}{Array}
					\SetKwData{EdgeBeforeJ}{EdgeBeforeJ}
					\SetKwData{Minimum}{Minimum}
					$\Minimum \gets \infty$ \;
					\For{$i \gets 1$ \KwTo $n - 1$}{
						$\Array[i] \gets (d_{V_i j} + d_{j V_{i + 1}}) - d_{V_i
							V_{i + 1}} $ \;
						\If{$\Array[i] < \Minimum$}{
							$\Minimum \gets \Array[i]$ \;
							$\EdgeBeforeJ \gets i$ \;
						}
					}
					Insert $j$ after \EdgeBeforeJ in $V$ \;
					\Return $V$ \;
				\end{algorithm}

				%TODO ask Kent about recurrence relation

			\item $C(S, 1)$ should be $\infty$ for all subsets $S$ with $1 \in
				S$ and $|S| > 1$ because this would be the minimal length path
				starting at 1, ending at 1, and traversing all cities in $S$
				\textit{exactly once}. To both start and end at 1 is to traverse
				1 twice, making this input invalid. An invalid minimum path must
				be longer than any valid minimum path, so it should be $\infty$.
				$C(\{ 1 \}, 1)$ is the minimum path from 1 to 1, visiting 1
				exactly once. The shortest path from any node to itself is zero,
				so $C(\{ 1 \}, 1) = 0$.

			\item The Held--Karp Algorithm solves the Traveling Salesman Problem
				in $O(n^2 2^n)$ time. This algorithm contains three nested
				\texttt{for} loops. The first runs $n$ times, the second runs
				for each subset of $S$, and the third runs for each vertex in
				that subset. There are $2^n$ subsets of $S$, so in the worst
				case this is $O(n \times 2^n \times n)$, which is $O(n^2 2^n)$.
				For a proof of this algorithm's correctness, see
				\cite{held1962dynamic}.

			\item $b^n = b \times b \times b \dots \times b$, whereas $n! = 1
				\times 2 \times 3 \dots \times n$. As $n$ grows to be much
				larger than $b$, the exponential sequence will still be
				multiplying by $b$ while the factorial sequence will be
				multiplying by increasingly larger and larger numbers. For
				sufficiently large $n$, $n!$ will outpace $b^n$ because $n$ can
				grow to be a larger multiplier.

			\item %TODO implement
		\end{enumerate}
\end{enumerate}

\printbibliography

\end{document}
