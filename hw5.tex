\documentclass[a4paper, 10pt]{article}

\usepackage[utf8]{inputenc} % UTF-8 support

\title{Analysis of Algorithms Homework 5}
\author{Joseph Petitti}
\date{\today}

\begin{document}

\maketitle

\begin{enumerate}
	\item First we show that this problem, \texttt{DoubleSAT}, is in
		$\mathsf{NP}$. Given $x \ge 2$ assignments $a_1, a_2, \dots , a_x$ you
		can simply evaluate $\phi$ with $a_1$ and $a_2$ to verify whether they
		satisfy $\phi$. Evaluating $\phi$ takes polynomial time. Therefore
		\texttt{DoubleSAT} is in $\mathsf{NP}$.

		Next, we show that SAT is reducible to \texttt{DoubleSAT}. Given an
		input formula $\phi(x_1, \dots, x_n)$ return a formula $\phi'$ with a
		new variable $y$ introduced:
		
		$$\phi'(x_1, \dots, x_n, y) = \phi(x_1, \dots, x_n) \land (y \lor \bar
		y)$$ 

		If $\phi(x_1, \dots, x_n) \in \mathrm{SAT}$, then $\phi$ has at least
		one satisfying assignment. Therefore, $\phi'$ has at least two
		satisfying assignments, one where $y$ is true and one where $y$ is
		false.

		If $\phi(x_1, \dots, x_n) \notin \mathrm{SAT}$, then $\phi$ has no
		satisfying assignment. Therefore, $\phi'$ clearly also has no
		satisfying assignments, because the first clause $\phi(x_1, \dots, x_n)$
		is always false.

		Therefore, SAT $\le_p$ \texttt{DoubleSAT}. Because it is in
		$\mathsf{NP}$ and an NP-complete problem (SAT) reduces to it,
		\texttt{DoubleSAT}is NP-complete.

	\item First we show that this problem, \texttt{DominatingSet}, is in
		$\mathsf{NP}$. We can easily verify that a set of vertices $S \subseteq
		V$ we can verify whether or not it is a dominating set of $G$ with size
		$\le D$ in polynomial time. We can simply check that the size of $S \le
		D$ and iterate over the $v \in V$ to check whether $v$ is a member of
		$S$ or any of its neighbors are members of $S$ in $O(n^2)$ time.
		Therefore \texttt{DominatingSet} is in $\mathsf{NP}$.

		Next, we show that vector cover is reducible to \texttt{DominatingSet}.
		Given $G$ we need to make a graph $G'$ such that $G'$ has a dominating
		set of size at most $D'$ if and only if $G$ has a vertex cover of size at
		most $D$. To translate from $G$ to $G'$ in polynomial time, we insert a
		new vertex connected to the endpoints of each edge in $G$. That is,
		for each edge $e \in E$ where $u, v$ are the endpoints of $e$, we add a
		vertex $w$ to $G'$, where $w$ is connected to $u$ and $v$. We keep the
		original edge $u, v$ as well. Now every vertex in the vector cover of
		$G$ is also in the dominating set of $G'$. The exception is nodes in $G$
		with no edges incident to them. We will simply include these in the
		dominating set, $D' = D + i$, where $i$ is the number of vertexes $v \in
		V$ such that no edge $e \in E$ is incident to $v$.

		Now we prove that $G'$ has a dominating set of size at most $D'$ if and
		only if $G$ has a vector cover of size at most $D$.
\end{enumerate}

\end{document}

