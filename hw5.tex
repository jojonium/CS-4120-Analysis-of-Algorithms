\documentclass[a4paper, 10pt]{article}

\usepackage[utf8]{inputenc} % UTF-8 support
\usepackage[]{algorithm2e}

\title{Analysis of Algorithms Homework 5}
\author{Joseph Petitti}
\date{\today}

\SetKwProg{Fn}{Function}{}{end}

\begin{document}

\maketitle

\begin{enumerate}
	\item First we show that this problem, \texttt{DoubleSAT}, is in
		$\mathsf{NP}$. Given $x \ge 2$ assignments $a_1, a_2, \dots , a_x$ you
		can simply evaluate $\phi$ with $a_1$ and $a_2$ to verify whether they
		satisfy $\phi$. Evaluating $\phi$ takes polynomial time. Therefore
		\texttt{DoubleSAT} is in $\mathsf{NP}$.

		Next, we show that SAT is reducible to \texttt{DoubleSAT}. Given an
		input formula $\phi(x_1, \dots, x_n)$ return a formula $\phi'$ with a
		new variable $y$ introduced:

		$$\phi'(x_1, \dots, x_n, y) = \phi(x_1, \dots, x_n) \land (y \lor \bar
		y)$$ 

		If $\phi(x_1, \dots, x_n) \in \mathrm{SAT}$, then $\phi$ has at least
		one satisfying assignment. Therefore, $\phi'$ has at least two
		satisfying assignments, one where $y$ is true and one where $y$ is
		false.

		If $\phi(x_1, \dots, x_n) \notin \mathrm{SAT}$, then $\phi$ has no
		satisfying assignment. Therefore, $\phi'$ clearly also has no
		satisfying assignments, because the first clause $\phi(x_1, \dots, x_n)$
		is always false.

		Therefore, SAT $\le_p$ \texttt{DoubleSAT}. Because it is in
		$\mathsf{NP}$ and an NP-complete problem (SAT) reduces to it,
		\texttt{DoubleSAT} is NP-complete.

	\item First we show that this problem, \texttt{DominatingSet}, is in
		$\mathsf{NP}$. We can easily verify that a set of vertices $S \subseteq
		V$ we can verify whether or not it is a dominating set of $G$ with size
		$\le D$ in polynomial time. We can simply check that the size of $S \le
		D$ and iterate over the $v \in V$ to check whether $v$ is a member of
		$S$ or any of its neighbors are members of $S$ in $O(n^2)$ time.
		Therefore \texttt{DominatingSet} is in $\mathsf{NP}$.

		Next, we show that the vertex cover problem (\texttt{VertexCover}) is
		reducible to \texttt{DominatingSet}.  Given $G$ we need to make a graph
		$G'$ such that $G'$ has a dominating set of size at most $D'$ if and
		only if $G$ has a vertex cover of size at most $D$. To translate from
		$G$ to $G'$ in polynomial time, we insert a new vertex connected to the
		endpoints of each edge in $G$. That is, for each edge $e \in E$ where
		$u, v$ are the endpoints of $e$, we add a vertex $w$ to $G'$, where $w$
		is connected to $u$ and $v$. We keep the original edge $u, v$ as well.
		Now every vertex in the vertex cover of $G$ is also in the dominating
		set of $G'$. The exception is nodes in $G$ with no edges incident to
		them. We will simply include these in the dominating set, $D' = D +
		|V_I|$, where $I$ is the number of vertexes $v \in V$ such that no edge
		$e \in E$ is incident to $v$.

		Now we prove that $G'$ has a dominating set of size at most $D'$ if and
		only if $G$ has a vertex cover of size at most $D$.

		\begin{itemize}
			\item $( \Rightarrow )$ Assume $G$ has a vertex cover $S$ of size at
				most $D$. $S' = S \cup V_I$ is a dominating set for $G'$.
				First, all vertices with no edges incident to them ($V_I$) are
				already in $S'$. Second, each new vertex $w$ that was added when
				we constructed $G'$ corresponds to an edge $(u, v)$ in $G$,
				which means that either $u$ or $v$ is in the vertex cover $S$,
				and $w$ is dominated by the same vertex in $S'$. All the other
				original vertices $v \in V$ incident to at least one edge in $E$
				are either in $S$ or adjacent to a vertex in $S$, because $v$ is
				covered by the vertex cover on $G$. Either way, it is adjacent
				to a vertex that is now in $S'$, and is dominated. Therefore
				$G'$ has a dominating set $S'$ of size at most $D' = D + |V_I|$.

			\item $( \Leftarrow )$ Now assume $G'$ has a dominating set $S'$ of
				size $D' = D + |V_I|$. If any special vertex $w_{uv}$ that was
				added when constructing $G'$ is in $S'$ (and therefore is not in
				$G$), replace it with $u$ in $S''$. Remove any isolated vertex in
				$V_I$ from $S''$, because it will not be in the vertex cover.
				$S''$ is now a vertex cover in of $G$. If any edge $(u, v)$ in
				$G$ was not covered by $S''$, the special vertex $w_{uv}$ would
				not be adjacent to any vertex of $S'$ in $G$, contradicting the
				claim that $S'$ is a dominating set of $G'$. Therefore $G$ has a
				vertex cover $S''$ of size at most $D$.
		\end{itemize}

		Therefore \texttt{DominatingSet} $\ge_p$ \texttt{VertexCover}.  Because
		it is in $\mathsf{NP}$ and is reducible to a NP-complete problem
		(\texttt{VertexCover}), \texttt{DominatingSet} is NP-complete.

	\item \begin{enumerate}
			\item First, we show that this problem, \texttt{FeedbackSet}, is in
				$\mathsf{NP}$. Given a set of vertices $S$ and a graph $G$, we
				can easily verify that $S$ is a feedback vertex set by deleting
				each vertex in $S$ and any edges containing those vertices from
				$G$, then running a depth-first search on each vertex of the
				resulting graph to determine whether $G' = (V \setminus S, E')$
				contains a cycle, where $G'$ is the where $E'$ is the set of all
				edges in $E$ that do not have a vertex in $S$.

				Next, we show that the vertex cover problem
				(\texttt{VertexCover}) is reducible to \texttt{FeedbackSet}.
				Given a graph $G$ and positive integer $k$, we need to make a
				graph $G' = (V', E')$ and positive integer $k'$ such that $G'$
				has a feedback vertex set of size at most $k'$ if and only if
				$G$ has a vertex cover of size at most $k$. To do this we need
				to translate every edge in $G$ to a cycle in $G'$. For each edge
				$(u, v) \in E$, add a vertex $w$ to $V'$. Add an edge $(u, w)$
				and an edge $(w, v)$ to $E'$. We will take $k' = k$.

				\begin{itemize}
					\item $( \Rightarrow )$ Assume $G$ has a vertex cover $S$ of
						size $k$. Removing every vertex in $S$ from $V'$ will
						break all the newly added triangles in $G'$, since each
						of those triangles contains an edge connected to a
						vertex in $S$. Any other cycle $C$ in $G'$ must also be
						in $G$, and an edge in $C$ must be incident to a vertex
						in $S$. Therefore, the vertex cover $S$ in $G$ is also a
						feedback vertex set in $G'$, so $G'$ has a feedback
						vertex set of size at most $k$.
					\item $( \Leftarrow )$ Assume $G'$ has a feedback vertex set
						$S'$ of size $k$. If any special vertex $w = (u, v)$
						that is not in $V$ is in $S'$, replace it with $u$ in
						$S''$.  Because we constructed $G'$ by replacing every
						edge in $G$ with a triangle, every edge $e \in E$ is
						part of a cycle. Since we replaced the newly added nodes
						$w_{uv}$, for every edge $e = (u, v) \in E$ at least one
						of $u, v$ must be in $S''$ in order to break the $\{ u,
						v, w \}$ cycle. Therefore every edge in $E$ is covered
						by a vertex in $S''$, so $G$ has a vertex cover of size
						at most $k$.
				\end{itemize}

			\item % TODO 3B
		\end{enumerate}

	\item If a graph $G$ has an independent set $S$ of size greater than $k$,
		then it also has an independent set of size exactly $k$, because you can
		simply remove vertices from $S$ until there are $k$ left and it will
		still be an independent set. See the algorithm below.

		\begin{function}[h]
			\SetKwFunction{FRecurs}{IndependentSet}
			\SetKwFunction{size}{size}
			\Fn{\FRecurs {$G, k$}}{
				\If{$\size{G} = k$}{
					\tcp{no more vertices can be removed}
					\Return $G$ \;
				}
				\BlankLine
				\Repeat{$\neg A(G', k)$}{
					$v \gets $ randomly chosen vertex in $G$ \;
					$V' \gets V \setminus v$ \;
					$G' \gets (V', E)$ \;
				}
				\BlankLine
				\Return \FRecurs {$G', k$} \;
			}
			\BlankLine
			\eIf{$A(G, k)$}{
				\Return \FRecurs {$G, k$} \;
			}{
				\Return ``No independent set of size at least $k$ exists'' \;
			}

		\end{function}
\end{enumerate}

\end{document}

