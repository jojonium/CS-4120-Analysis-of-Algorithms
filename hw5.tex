\documentclass[a4paper, 10pt]{article}

\usepackage[utf8]{inputenc} % UTF-8 support

\title{Analysis of Algorithms Homework 5}
\author{Joseph Petitti}
\date{\today}

\begin{document}

\maketitle

\begin{enumerate}
	\item First we show that this problem, \texttt{DoubleSAT}, is in
		$\mathsf{NP}$. Given $x \ge 2$ assignments $a_1, a_2, \dots , a_x$ you
		can simply evaluate $\phi$ with $a_1$ and $a_2$ to verify whether they
		satisfy $\phi$. Evaluating $\phi$ takes polynomial time. Therefore
		\texttt{DoubleSAT} is in $\mathsf{NP}$.

		Next, we show that SAT is reducible to \texttt{DoubleSAT}. Given an
		input formula $\phi(x_1, \dots, x_n)$ return a formula $\phi'$ with a
		new variable $y$ introduced:
		
		$$\phi'(x_1, \dots, x_n, y) = \phi(x_1, \dots, x_n) \land (y \lor \bar
		y)$$ 

		If $\phi(x_1, \dots, x_n) \in \mathrm{SAT}$, then $\phi$ has at least
		one satisfying assignment. Therefore, $\phi'$ has at least two
		satisfying assignments, one where $y$ is true and one where $y$ is
		false.

		If $\phi(x_1, \dots, x_n) \notin \mathrm{SAT}$, then $\phi$ has no
		satisfying assignment. Therefore, $\phi'$ clearly also has no
		satisfying assignments, because the first clause $\phi(x_1, \dots, x_n)$
		is always false.

		Therefore, SAT $\le_p$ \texttt{DoubleSAT}. Because it is in
		$\mathsf{NP}$ and an NP-complete problem (SAT) reduces to it,
		\texttt{DoubleSAT} is NP-complete.

	\item First we show that this problem, \texttt{DominatingSet}, is in
		$\mathsf{NP}$. We can easily verify that a set of vertices $S \subseteq
		V$ we can verify whether or not it is a dominating set of $G$ with size
		$\le D$ in polynomial time. We can simply check that the size of $S \le
		D$ and iterate over the $v \in V$ to check whether $v$ is a member of
		$S$ or any of its neighbors are members of $S$ in $O(n^2)$ time.
		Therefore \texttt{DominatingSet} is in $\mathsf{NP}$.

		Next, we show that the vector cover problem (\texttt{VectorCover}) is
		reducible to \texttt{DominatingSet}.  Given $G$ we need to make a graph
		$G'$ such that $G'$ has a dominating set of size at most $D'$ if and
		only if $G$ has a vertex cover of size at most $D$. To translate from
		$G$ to $G'$ in polynomial time, we insert a new vertex connected to the
		endpoints of each edge in $G$. That is, for each edge $e \in E$ where
		$u, v$ are the endpoints of $e$, we add a vertex $w$ to $G'$, where $w$
		is connected to $u$ and $v$. We keep the original edge $u, v$ as well.
		Now every vertex in the vector cover of $G$ is also in the dominating
		set of $G'$. The exception is nodes in $G$ with no edges incident to
		them. We will simply include these in the dominating set, $D' = D +
		|V_I|$, where $I$ is the number of vertexes $v \in V$ such that no edge
		$e \in E$ is incident to $v$.

		Now we prove that $G'$ has a dominating set of size at most $D'$ if and
		only if $G$ has a vector cover of size at most $D$.
		
		\begin{itemize}
			\item $( \Rightarrow )$ Assume $G$ has a vertex cover $S$ of size at
				most $D$. $S' = S \cup V_I$ is a dominating set for $G'$.
				First, all vertices with no edges incident to them ($V_I$) are
				already in $S'$. Second, each new vertex $w$ that was added when
				we constructed $G'$ corresponds to an edge $(u, v)$ in $G$,
				which means that either $u$ or $v$ is in the vertex cover $S$,
				and $w$ is dominated by the same vertex in $S'$. All the other
				original vertices $v \in V$ incident to at least one edge in $E$
				are either in $S$ or adjacent to a vertex in $S$, because $v$ is
				covered by the vertex cover on $G$. Either way, it is adjacent
				to a vertex that is now in $S'$, and is dominated. Therefore
				$G'$ has a dominating set $S'$ of size at most $D' = D + |V_I|$.

			\item $( \Leftarrow )$ Now assume $G'$ has a dominating set $S'$ of
				size $D' = D + |V_I|$. If any special vertex $w_{uv}$ that was
				added when constructing $G'$ is in $S'$ (and therefore is not in
				$G$), replace it with $u$ in $S''$. Remove any isolated vertex in
				$V_I$ from $S''$, because it will not be in the vertex cover.
				$S''$ is now a vertex cover in of $G$. If any edge $(u, v)$ in
				$G$ was not covered by $S''$, the special vertex $w_{uv}$ would
				not be adjacent to any vertex of $S'$ in $G$, contradicting the
				claim that $S'$ is a dominating set of $G'$. Therefore $G$ has a
				vertex cover $S''$ of size at most $D$.
		\end{itemize}

		Therefore \texttt{DominatingSet} $\ge_p$ \texttt{VectorCover}.  Because
		it is in $\mathsf{NP}$ and is reducible to a NP-complete problem
		(\texttt{VectorCover}), \texttt{DominatingSet} is NP-complete.
\end{enumerate}

\end{document}

