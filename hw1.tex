\documentclass[a4paper, 10pt]{article}

\usepackage[utf8]{inputenc} % UTF-8 support
\usepackage{listings} % for code listings

\title{Analysis of Algorithms Homework 1}
\author{Joseph Petitti}
\date{\today}

\begin{document}

\maketitle

\begin{enumerate}
	\item Functions ordered by ascending order of growth rate (\textit{i.e.} the
		last item grows the fastest):
		\begin{itemize}
			\item $f_7(n) = n^{1 / \log n}$
			\item $f_2(n) = \sqrt{2n}$
			\item $f_3(n) = 10 + n$
			\item $f_6(n) = n^2 \log (n)$
			\item $f_1(n) = n^{2.5}$
			\item $f_4(n) = 10^{n}$
			\item $f_5(n) = 10^{2n}$
		\end{itemize}

	\item Trees are connected and acyclic by definition. If $G$ has fewer than
		$n - 1$ edges then it cannot be connected. If $G$ has more than $n - 1$
		edges:
		\begin{itemize}
			\item Suppose $G$ has more then $n - 1$ edges but  is not a tree.
			\item Because $G$ is connected but not a tree, it must contain a
				cycle.
			\item Remove an edge from the cycle. We are left with a connected
				graph with $n-2$ edges, which is impossible.
		\end{itemize}

	\item \textbf{Efficient Algorithm}: Breadth-first search (assume the graph
		is given in the adjacency list representation):

\begin{lstlisting}
Choose a starting node s
Set Discovered[s] = true
Set Discovered[v] = false for all other v in G
Initialize L[0] to consist of the single element s
Set the layer counter i = 0
While L[i] is not empty
	Initialize an empty list L[i+1]
	For each node u in L[i]
		Consider each edge (u, v) incident to u
		If Discovered[v] = false then
			Set Discovered[v] = true
			Add v to the list L[i + 1]
		Else
			Return true
		Endif
	EndFor
	Increment the layer counter i by 1
endWhile
Return false
\end{lstlisting}

		\textbf{Proof of Correctness}: This algorithm, which is essentially just
		breadth-first search, goes through each node in $G$ by ``layer,''
		where each layer $n$ is the set of nodes exactly $n$ distance away from
		starting node $s$. This algorithm always moves ``down'' the graph, if
		you encounter a node twice it means there is a loop.

		\textbf{Analysis of Running Time}: This algorithm runs in $O(m + n)$
		time, because it is really just breadth-first search.

	\item \begin{enumerate}
			\item
				Assuming there is a node $v \in V$ with $\textrm{degree} (v) =
				1$ (\textit{i.e.} $v$ is a leaf node):

				\begin{itemize}
					\item Because the graph is connected, the one node that $v$
						is connected to is also connected to the rest of the
						graph.
					\item Therefore if you remove $v$ the rest of the graph will
						still be connected.
					
				\end{itemize}

				If there is no node $v \in V$ with $\textrm{degree} (v) = 1$
				(\textit{i.e.} no leaf nodes):

				\begin{itemize}
					\item If all nodes $v \in V$ have $\textrm{degree} (v) > 1$
						then the graph must contain a cycle.
					\item In a set of nodes $S$, where $S$ is a cycle, removing
						a single node will result in a sequence of connected
						nodes.   
					\item Therefore, you can remove any node from the cycle and
						the graph will still be connected.
				\end{itemize}

			\item Proof that for every strongly connected directed graph $G =
				(V, E)$ there exists a vertex $v \in V$ such that removing $v$
				from $G$ results with a strongly connected graph.

				\begin{itemize}
					\item In a strongly connected graph, every node is reachable
						from every other node by definition.
					\item Let $S$ be the set of set of all vertices in $G$
						except for $v$.
					\item All nodes in $S$ are reachable by all other nodes in
						$S$, because $S$ is a subset of $G$.
					\item If $v$ is removed from the $G$, you are left with $S$,
						which is still strongly connected.
				\end{itemize}
		\end{enumerate}


	\item Proof by contradiction:
		\begin{itemize}
			\item Suppose there exists a cut $(S, T)$ with no edge $e \in E$
				that crosses it.
			\item Therefore no edge connects a node in $S$ with a node in $T$.
			\item If none of the nodes in $S$ are connected to any node in $T$
				then there must exist some node $s \in S$ that is not connected
				to some node $t \in T$.
			\item Therefore $G$ is not connected.
		\end{itemize}
\end{enumerate}

\end{document}

