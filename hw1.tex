\documentclass[a4paper, 10pt, american]{article}

\usepackage[utf8]{inputenc} % UTF-8 support
%\usepackage[margin=1in]{geometry} % set 1in margins

\title{Analysis of Algorithms Homework 1}
\author{Joseph Petitti}
\date{\today}

\begin{document}

\maketitle

\begin{enumerate}
	\item Functions ordered by ascending order of growth rate (\textit{i.e.} the
		last item grows the fastest):
		\begin{itemize}
			\item $f_7(n) = n^{1 / \log n}$
			\item $f_2(n) = \sqrt{2n}$
			\item $f_3(n) = 10 + n$
			\item $f_6(n) = n^2 \log (n)$
			\item $f_1(n) = n^{2.5}$
			\item $f_4(n) = 10^{n}$
			\item $f_5(n) = 10^{2n}$
		\end{itemize}

	\item Trees are connected and acyclic by definition. If $G$ has fewer than
		$n - 1$ edges then it cannot be connected. If $G$ has more than $n - 1$
		edges:
		\begin{itemize}
			\item Suppose $G$ has more then $n - 1$ edges but  is not a tree.
			\item Because $G$ is connected but not a tree, it must contain a
				cycle.
			\item Remove an edge from the cycle. We are left with a connected
				graph with $n-2$ edges, which is impossible.
		\end{itemize}

	\item \textbf{Efficient Algorithm}: Do a breadth-first search of $G$,
		remembering each vertex you have seen. If you ever see a vertex twice,
		$G$ contains a cycle

		\textbf{Proof of Correctness}: 

		\textbf{Analysis of Running Time}:

	\item \begin{enumerate}
			\item
			\item
		\end{enumerate}


	\item Proof by contradiction:
		\begin{itemize}
			\item Suppose there exists a cut $(S, T)$ with no edge $e \in E$
				that crosses it.
			\item Therefore no edge connects a node in $S$ with a node in $T$.
			\item If none of the nodes in $S$ are connected to any node in $T$
				then there must exist some node $s \in S$ that is not connected
				to some node $t \in T$.
			\item Therefore $G$ is not connected.
		\end{itemize}
\end{enumerate}

\end{document}

