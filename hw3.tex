\documentclass[a4paper, 10pt]{article}

\usepackage[utf8]{inputenc} % UTF-8 support
\usepackage{amsmath}
\usepackage[]{algorithm2e}

\title{Analysis of Algorithms Homework 3}
\author{Joseph Petitti}
\date{\today}

\begin{document}

\maketitle

\begin{enumerate}
	\item \textbf{Algorithm}: Let $w(v_i)$ be the weight of vertex $i$.

		Base case:

		$$ \textrm{Maximum weight } =
		\begin{cases}
			0 & \textrm{if } n = 0 \\
			w(v_1) & \textrm{if } n = 1
		\end{cases} $$

		Store this value in $A[0]$. When looking at vertex $v_i$, the memoized
		value, $A[i] = \max ( A[i - 1], A[i - 2] + w(v_i))$. Then, if $A[i] \ne
		A[i - 1]$ (\textit{i.e.} we chose the second option), $v_i$ is in the
		optimal solution. After looking at every vertex, $A[n]$ is the maximal
		weight.

		\textbf{Proof of correctness by induction}: Base case: when $n = 0$
		$A[0] = 0$, which is the optimal solution (the maximal weight of a set
		of 0 vertices is~0).

		Inductive hypothesis: for all values of $n$ such that $n > 0$, $A[n]$
		computed by the above algorithm is the maximal weight of an independent
		set of $P$.

		Inductive step: $n = i + 1$. When the algorithm reaches a vertex $v_i$,
		either $v_i$ belongs to the optimal solution or it does not. At vertex
		$i$, 
		
		\[ A[i] =
		\begin{cases}
			A[i - 1] & \textrm{if } v_i \textrm{ is not in the optimal
				solution} \\
			A[i -2] + w(v_i) & \textrm{if } v_i \textrm{ is in the optimal
				solution}
		\end{cases} \]

		The algorithm chooses the maximum of the two cases, so it always chooses
		the maximum possible value.

		\textbf{Analysis of running time}: This algorithm runs is $O(n)$,
		because it needs to only examine each vertex in $P$ once, performing
		constant time operations on each one. Memoization of values that would
		need to be calculated multiple times in the brute force approach allow
		us to only calculate each value only once.

		\newpage

	\item Let $\texttt{substr}(i, j)$ be the substring of $r$ from character $i$
		to character $j$. Initialize an array $A$ of length $n$. Each value
		$A[i]$ will indicate whether $\texttt{substr}(1, i)$ is a concatenation
		of strings from $D$.

		\IncMargin{1em}
		\begin{algorithm}[h]
			\SetKwFunction{substr}{substr}
			\For{$j \gets 1$ \KwTo $n$}{
				\eIf{$\substr(1, j)$ belongs to $D$}{
					$A[j] \gets$ true \;
				}{
					\For{$k \gets 1$ \KwTo $j - 1$}{
						$A[j] \gets (A[k] \land (\substr(k + 1, j)
						\textrm{ belongs to } D))$ \;
					}
				}
			}
			\Return{A[n]}
			\label{alg:prob2}
		\end{algorithm}

		This algorithm is $O(n^2)$ because in the worst case the inner loop runs
		$n$ times, and the outer loop runs $n$ times.

	\item \textbf{Algorithm}: Let $n$ be the number of items, $w(i)$ be the
		weight of item $i$, and $v(i)$ be the value of item $i$. Initialize a
		two-dimensional array $A[W_1][W_2]$, with all values initialized to 0.

		\begin{algorithm}[h]
			\SetKwData{maxValue}{maxValue}
			\maxValue $\gets 0$\;
			\For{$i \gets n - 1$ \KwTo $1$}{
				\For{$j \gets 1$ \KwTo $W_1$}{
					\For{$k \gets 1$ \KwTo $W_2$}{
						\If(\tcp*[f]{put $i$ in knapsack 1}){$j > w(i)$}{
							$A[j][k] \gets \max(A[j][k], v(i) + A[j - w(i)][k]$
							\;
							\maxValue $\gets \max(\maxValue, A[j][k])$\;
						}
						\If(\tcp*[f]{put $i$ in knapsack 2}){$k > w(i)$}{
							$A[j][k] \gets \max(A[j][k], v(i) + A[j][k - w(i)]$
							\;
							\maxValue $\gets \max(\maxValue, A[j][k])$\;
						}
					}
				}
			}
			\Return{\maxValue}
			\label{alg:prob3}
		\end{algorithm}

		To get the actual selected items can be calculated by tracing back
		through the array using the algorithm described on page 271 of
		``Algorithm Design'' by Jon Kleinberg and Éva Tardos.

		\textbf{Proof of correctness}: This algorithm calculates all possible
		combinations of items in knapsacks (it merely avoids the cost associated
		with brute-force search by avoiding recalculating values). It chooses
		the maximum among all possible combinations, so it must be correct.

		\textbf{Analysis of running time}: This algorithm is $O(n W_1 W_2)$
		because it contains three nested loops, the first running $n$ times, the
		second running $W_1$ times, and the third running $W_2$ times.

		\newpage

	\item \textbf{Algorithm}: Initialize a two-dimensional array $A[n][n]$.

		\begin{algorithm}[h]
			\SetKwData{maxValue}{maxValue}
			\SetKwData{maxI}{optI}
			\SetKwData{maxJ}{optJ}
			\maxValue $\gets -\infty$ \;
			\For{$i \gets 1$ \KwTo $n$}{
				$A[i][i] \gets a_i$ \;
				\If{$A[i][i] > \maxValue$}{
					$\maxValue \gets A[i][i]$ \;
					$\maxI \gets i$ \;
					$\maxJ \gets i$ \;
				}
				\For{$j \gets i + 1$ \KwTo $n$}{
					$A[i][j] = A[i][j - 1] + a_i$ \;
					\If{$A[i][j] > \maxValue$}{
						$\maxValue \gets A[i][j]$ \;
						$\maxI \gets i$ \;
						$\maxJ \gets j$ \;
					}
				}
			}
			\Return{\maxI, \maxJ}
			\label{alg:prob4}
		\end{algorithm}

		\textbf{Proof of correctness by induction}: Base case: when $n$ is 1,
		the above algorithm will return $i = 1, j = 1$, which is obviously the
		optimal solution. The maximum subset of a set with one element is just
		the one element.

		Inductive hypothesis: For all values of $n > 1$, the values $i, j$
		returned by the above algorithm will represent the maximum sequential
		subset of $S$.

		Inductive step: $n = i + 1$. The algorithm simply considers adding $a_i$
		to the correctly computed maximal sequential subset for $n = i$, and
		takes this as the new optimal solution if it is better.

		\textbf{Analysis of running time}: This algorithm is $O(n^2)$, because
		it contains two nested loops that each run $n$ or fewer times.

\end{enumerate}

\end{document}

