\documentclass[a4paper, 10pt]{article}

\usepackage[utf8]{inputenc} % UTF-8 support
\usepackage{amsmath}

\title{Analysis of Algorithms Homework 3}
\author{Joseph Petitti}
\date{\today}

\begin{document}

\maketitle

\begin{enumerate}
	\item \textbf{Algorithm}:

		Let $w(v_i)$ be the weight of vertex $i$.

		Base case:

		$$ \textrm{Maximum weight } =
		\begin{cases}
			0 & \textrm{if } n = 0 \\
			w(v_1) & \textrm{if } n = 1
		\end{cases} $$

		Store this value in $A[0]$. When looking at vertex $v_i$, the memoized
		value, $A[i] = \max ( A[i - 1], A[i - 2] + w(v_i))$. Then, if $A[i] \ne
		A[i - 1]$ (\textit{i.e.} we chose the first option), $v_i$ is in the
		optimal solution. After looking at every vertex, $A[n]$ is the maximal
		weight.

		\textbf{Proof of correctness by induction}:

		Base case: when $n = 0$ $A[0] = 0$, which is the optimal solution (the
		maximal weight of a set of 0 vertices is 0).

		Inductive hypothesis: for all values of $n$ such that $n > 0$, $A[n]$
		computed by the above algorithm is the maximal weight of an independent
		set of $P$.

		Inductive step: $n = i + 1$. When the algorithm reaches a vertex $v_i$,
		either $v_i$ belongs to the optimal solution or it does not. At vertex
		$i$, 
		
		\[ A[i] =
		\begin{cases}
			A[i - 1] & \textrm{if } v_i \textrm{ is not in the optimal
				solution} \\
			A[i -2] + w(v_i) & \textrm{if } v_i \textrm{ is in the optimal
				solution}
		\end{cases} \]

		\textbf{Analysis of running time}:
		
		This algorithm runs is $O(n)$, because it needs to only examine each
		vertex in $P$ once, performing constant time operations on each one.
		Memoization of values that would need to be calculated multiple times in
		the brute force approach allow us to only calculate each value only
		once.

	\item

\end{enumerate}

\end{document}

